\documentclass[a4paper,twoside,10pt]{article}
\usepackage[T1]{fontenc}
\usepackage[utf8]{inputenc}
\usepackage{lmodern}
\usepackage{parskip}
\usepackage[ngerman]{babel}
\usepackage{a4 wide}
\author{Matthias Meyer}
\title{Meine LaTeX-Übung}
\date{\today}

\begin{document}
\textbf{Unnummerierte Listen} Listen haben Punkte, Striche oder sonstige Zeichen als Anstriche
 \begin{itemize}
    \item Punkt
    \item Noch ein Punkt
    \begin{itemize}
        \item Unterpunkt
        \item Weiterer Unterpunkt
        \begin{itemize}
            \item Unterunterpunkt
        \end{itemize}        
    \end{itemize}
    \item Ein weiterer Punkt
 \end{itemize}

 \textbf{Nummerierte Listen} Listen haben Punkte, Striche oder sonstige Zeichen als Anstriche

 \begin{enumerate}
     \item Erster Punkt
     \begin{enumerate}
         \item Unterpunkt
         \item Unterpunkt
         \begin{enumerate}
            \item Unterunterpunkt
            \begin{enumerate}
                \item Unterunterunterpunkt
                \item Unterunterunterpunkt
            \end{enumerate}
            \item Unterunterpunkt
         \end{enumerate}
     \end{enumerate}
     \item Zweiter Punkt
     \item Dritter Punkt
     \begin{enumerate}
         \item Noch ein Unterpunkt
         \item Noch ein Unterpunkt
     \end{enumerate}
 \end{enumerate}

 \textbf{Beschreibende Listen} haben einen individuellen Text am Anfang, wie z.B. diese Liste
 
 \begin{description}
     \item[Begriff] Erklärung
     \item[Noch ein Begriff] Weitere Erklärung 
 \end{description}
\end{document}