\documentclass[a4paper,twoside,10pt]{article}
\usepackage[T1]{fontenc}
\usepackage[utf8]{inputenc}
\usepackage{lmodern}
\usepackage{parskip}
\usepackage[ngerman]{babel}
\usepackage{a4 wide}

\begin{document}
Leerzeichen und Zeilenumbrüche erzeugen nicht immer unbedingt Leerzeichen und Zeilenumbrüche im Ausgabedokument, wie man zunächst vermuten möchte. Eine Folge von mehreren Leerzeichen wird als Leerzeichen angesehen. D.h., zwischen Wörtern können beliebig viele Leerzeichen stehen und trotzdem wird der Zwischenraum nicht größer. Auch ein einzelner Zeilenumbruch ist nur ein Leerzeichen.

Mehrere Zeilenumbrüche in Folge leiten jedoch einen neuen Absatz ein. Der Abstand zwischen den Absätzen ist aber immer gleich, egal wie viele Zeilenumbrüche ihr Eingabedokument an dieser Stelle hat.

Mit diesem Mechanismus können Sie folglich Ihren Text im Eingabefile nach Belieben umbrechen. Doppelte Leerzeichen können sich ebenfalls nicht einschleichen. Probieren sie das Hinzufügen zusätzlicher Leerzeichen und Zeilenumbrüche einfach an diesem Beispieltext aus.
\end{document}